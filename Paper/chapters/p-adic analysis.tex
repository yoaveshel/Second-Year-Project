\chapter{$p$-adic Analysis}
The field of $p$-adic analysis is vast, and many things from real analysis translate to the $p$-adic world, with unique and surprising property.
We discuss here only one interesting application of $p$-adic number, called Newton polygons. 
This method assigns a polygon to every polynomial and power series with coefficients in $\Cp$, which tells us a lot about the roots of that polynomial and the radius of converges of the power series.
\section{Newton polygons for polynomials}
In this section we shall denote $f(X)$ as a polynomial of the form $f(X) = 1 + \sum_{i=1}^{n} a_i X^i$ with $a_i \in \C_p$. Since $\C_p$ is algebraically closed, every polynomial $f(X)$ in $\C_p[X]$ has a root and we will discover the behavior of said roots using Newton polygon.

Using the coefficients of $f(X)$, plot the following points 
\[(0,0), \left(1, \valp{(a_1)}\right), \left(1, \valp{(a_2)}\right),\dots,\left(1, \valp{(a_n)}\right)\] 
on a plane and if any $a_i = 0$, we simply ignore it. Then the Newton polygon can be constructed as follow: from the point $(0, 0)$ construct a vertical line through $(0, 0)$ and rotate it counter clockwise until it hits a second point then said line is a segment joining $(0, 0)$ and $(i, \valp{a_i})$. 
Repeat the same process for every other points to finish the construction. 
This generates the lower boundary of the convex hull of the above set of points.
For each segments in a Newton polygon, we need to pay attention these items:
\begin{enumerate}
    \item The slope of the segment.
    \item The length of the projection of the segment onto the $x$-axis.
    \item The values of $i$ at each vertices or breaks of the polygon.
\end{enumerate}
The reason why said items are special will become apparent soon. 
\begin{example} Let $f(X) = 1 + X^2 + 1/3 X^3 + 3X^4$ with $p = 3$. Then the points are $(0,0)$, $(2,0)$, $(3, -1)$, $(4, 1)$. 
\end{example}

\pgfplotsset{my style/.append style={axis x line=middle, axis y line=middle, xlabel={$x$}, ylabel={$y$} }}
\begin{figure}[H]
    \centering
    \begin{tikzpicture}
    \begin{axis}[my style, xtick={-1,0,...,4}, ytick={-1,0,...,2},xmin=0, xmax=4,  ymin=-1, ymax=1]
    \addplot[mark=*, only marks] coordinates {(0,0)(2,0)(3,-1)(4,1)};
    \addplot[mark=*] coordinates {(0,0)(3,-1)(4,1)};
    \end{axis}
    \end{tikzpicture}
    \label{fig:newton-polygon}
    \caption{The Newton polygon of $f(X)=1+X^2+1/3 X^3+3X^4\in\C_3[X]$}
\end{figure}

As it turns out, the slopes of the Newton polygon count the number of roots of a given absolute value. The following theorem is adapted from \cite[p. 97]{Koblitz_2012}
\begin{theorem}
Let $f(X) = (1 - X/\alpha_1)(1 - X/\alpha_2)...(1 - X/\alpha_n)$ be the factorization of $f(X)$ in terms of its roots in $\C_p$. Let $\lambda_i = \valp{(1/\alpha_i)}$. Then if $\lambda$ is a slope of the Newton polygon having length $l$, it follows that there are $l$ of $\lambda_i$ equal to $\lambda$ (counting multiplicities). 
\begin{proof}
Suppose that the roots $\alpha_i$ are arranged in such a way that $\lambda_1 \leq \lambda_2 \leq ... \lambda_n$. Assume that $\lambda_1 = \lambda_2 = ... = \lambda_r < \lambda_{r + 1}$ for $r \in \{2,3,...,n\}$. We first claim that the first segment of the polygon is the segment joining $(0,0)$ and $(r, r\lambda_1)$. Recall that each $a_i$ is expressed in terms of $1/\alpha_1$, $1/\alpha_2$, ..., $1/\alpha_n$ as the sum of all possible products of $i$ of the $1/\alpha$'s. Since $p$-adic valuation of such product is at least $i\lambda_i$, the same is true for $a_i$, so the point $(i, \valp{(a_i)})$ is on or above the point $(i, i\lambda_1)$.\\
\indent Now consider $a_r$. Of the various products of $r$ of the $1/\alpha$'s, exactly one has the valuation $r\lambda_1$, namely, the product $1/(\alpha_1 \alpha_2 ... \alpha_r)$. All of the other products have valuation $> r\lambda_1$ since we must include at least one of the $\lambda_{r+1},\lambda_{r+2},...,\lambda_n$. Thus, $a_r$ is a sum of something with valuation $r\lambda_1$ and something with valuation $> r\lambda_1$, so by a property of the non-Archimedean norm, $\valp{(a_r)} = r\lambda_1$. \\
\indent Suppose that $i > r$. In the same way as before, we see that all of the products of $i$ of the $1/\alpha$'s have valuation $> i\lambda_i$. Hence, $\valp{a_i} > i\lambda_i$. If we now think of how Newton polygon is constructed, we see that we have shown that its first segment is the line joining $(0,0)$ with $(r, r\lambda_i)$.\\
\indent If for $s \in \{1,2,..., n\}$, we have $\lambda_s < \lambda_{s + 1} = \lambda_{s + 2} = ... = \lambda_{r} < \lambda_{r + 1} < ... < \lambda_{n}$ with $r \in \{s, s+1, ..., n \}$, then the line joining $(s, \sum_{i=1}^{s} \lambda_i)$ to $(r, \sum_{i=1}^{s} \lambda_i + (r - s)\lambda_{s+1})$ is a segment of the Newton polygon can be proved analogously. 
\end{proof}
\end{theorem}
In other words, the slopes of the Newton polygon of $f(X)$ are the valuations of the reciprocal roots of $f(X)$ (counting multiplicities). 

\begin{example} Let $f(X) = 1 + 1/25 X^2 + 5X^3 + 25X^4 + 7X^5$ with $p = 5$. Then we have a set of points $(0,0)$, $(2,-2)$, $(3,1)$, $(4,2)$, $(5,0)$ with its Newton polygon

\begin{figure}[H]
    \centering
    \begin{tikzpicture}
    \begin{axis}[my style, xtick={0,1,...,5}, ytick={-2,-1,0,...,2},xmin=0, xmax=5,  ymin=-2, ymax=2]
    \addplot[mark=*, only marks] coordinates {(0,0)(2,-2)(3,1)(4,2)(5,0)};
    \addplot[mark=*] coordinates {(0,0)(2,-2)(5,0)};
    \end{axis}
    \end{tikzpicture}
    \label{fig:newton-polygon2}
    \caption{Newton polygon of $f(X)=1+1/25 X^2+5X^3+25X^4+7X^5\in\C_5[X]$}
\end{figure}
Observably, there are two segments of slopes $-1$, $2/3$ and lengths $2$, $3$ respectively. Thus, there are two and three roots with absolute value $5$ and $5^{-2/3}$ respectively. 
\end{example}


\section{Newton polygons for power series}
Having studied the Newton polygon for a polynomial, one of the natural next steps is to consider the Newton polygon of a power series $f(X) = 1 + \sum_{i=1}^{\infty} a_i X^i$ with $a_i \in \C_p$. 
The rules for the construction of the Newton polygon shall be revised as: start with the vertical half-line which is in negative part of the $y$-axis and rotate that line counter-clockwise until one of the following occurs:
\begin{enumerate}
    \item The line simultaneously hits infinitely many of the points plotted. In this case, stop and the construction is complete.
    \item The line reaches a position where it contains only one of our points but can be rotated no further without leaving behind some points. In this case, stop and the construction is complete.
    \item The line hits a finite number of the points. In this case, break the line at the last point that was hit and begin the whole construction again.
\end{enumerate}

\begin{example}
Let $f(X) = 1 + \sum_{i=1} p^i X^i$, Then the points are $(0,0)$, $(1,1)$, $(2,2)$, ... 
\begin{figure}[H]
    \centering
    \begin{tikzpicture}
    \begin{axis}[my style, xtick={0,1,...,6}, ytick={0,1,...,6},xmin=0, xmax=6,  ymin=0, ymax=6]
    \addplot[mark=*, only marks] coordinates {(0,0)(1,1)(2,2)(3,3)(4,4)(5,5)};
    \addplot[mark=*] coordinates {(0,0)(1,1)(2,2)(3,3)(4,4)(5,5)};
    \end{axis}
    \end{tikzpicture}
    \label{fig:newton-polygon-for-powerseries}
    \caption{Newton polygon of $1 + \sum_{i=1} p^i X^i$}
\end{figure}
\end{example}
Similar to that of a polynomial, the Newton polygon also gives us useful information about its zeros but for the purpose of this paper, we shall only look at its radius of convergence. 
We start by proving the following lemma, which is one of the many nice properties of non-Archimedean absolute values.


\begin{lemma}
Let $\langle a_k \rangle$ be a sequence in $\C_p$. Then $\sum_{k=0}^{\infty} a_k$ converges in $\C_p$ if and only if $\lim_{k \rightarrow \infty} a_k = 0$.
\begin{proof}
Since the first implication is a known result, we shall only show the converse. Suppose that $a_k \rightarrow 0$ as $k \rightarrow \infty$. Let $A_n := \sum_{k=0}^n a_k$ be the partial sum of the series then for any integers $m,n$ with $0 < m < n$ we have 
\[ \absp{A_n - A_m} = \absp{\sum_{k=m+1}^n a_k} \leq \max (\absp{a_{m+1}},...,\absp{a_n}) \rightarrow 0 \ \text{as} \ m,n \rightarrow \infty. \]
So the partial sums $A_n$ forms a Cauchy sequence, hence must converge to a limit in $\C_p$. 
\end{proof}
\end{lemma}

\begin{theorem}
Let $b$ be the least upper bound of all slopes of the Newton polygon of $f(X)$. Then the radius of convergence is $p^b$ (if $b$ is infinite then $f(X)$ converges on all of $\C_p$).
\begin{proof}
First let $x \in \C_p$ and $\absp{x} < p^b$, i.e., $\valp{(x)} > -b$. Let $\valp{(x)} = -b'$, where $b' < b$. Then $\valp{(a_i x_i)} = \valp{(a_i)} - ib'$. 
But it is clear that, sufficiently far out, the point $(i, \valp{(a_i)})$ lies arbitrarily far above $(i, b'i)$, in other words, $\valp{(a_i x^i)} \rightarrow \infty$, and $f(X)$ converges at $X = x$.\\
\indent Now let $\absp{x} > p^b$, i.e., $\valp{(x)} = -b' < -b$. Then we find in the same way that $\valp{(a_i x^i)} = \valp{(a_i)} - b'i$ is negative for infinitely many values of $i$. Thus $f(x)$ does not converge. 
\end{proof}
\end{theorem}
In other word, by drawing the Newton polygon and the line $y=\valp(a)x$ it is immediate that power series converges at $a$ if the Newton polygons lies above the line for large enough values of $i$.


